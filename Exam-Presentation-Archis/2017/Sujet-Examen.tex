\documentclass[twocolumn, a4]{article}
% \documentclass[a4]{article}
\usepackage{eurosym} % \euro{} \geneuro{} \geneuronarrow{} \geneurowide
\usepackage[french]{babel}
\usepackage[T1]{fontenc}
\usepackage[latin1]{inputenc}
\usepackage{xcolor}
\usepackage{a4}
% \usepackage{url}
\usepackage{sectsty}
\usepackage{hyperref}
%\usepackage[utf8]{inputenc}
%\usepackage{helvet}
\usepackage{graphicx}
\definecolor{RED}{rgb}{1.,.1,.1}
\newcommand{\TODO}[1]{\marginpar{\textcolor{RED}{#1}}}
\title{Examen Architecture des Processeurs}
\author{Henri-Pierre Charles \& Fr�d�ric Rousseau}

\begin{document}
\sffamily % http://www.tug.org/pracjourn/2006-1/schmidt/schmidt.pdf
\allsectionsfont{\sffamily}
\maketitle
% \tableofcontents{}

\section*{Domaine de l'examen}


Il faut r�aliser une pr�sentation d'une architecture contenant au
moins les �l�ments suivants :
\begin{itemize}
\item Les �l�ments de performance : Fr�quence d'horloge, Flops / Mips
  / Watt / ...
\item La hi�rarchie m�moire : Caches / Associativit� / Politique / D�bits
\item Pipeline:  profondeur, nombre d'UAL, sp�culatif, ...
\item Jeux d'instruction : Format d'instruction (taille, nombre
  registres, encodage), mode d'adressage, exemple comment�s
  d'assembleur (par exemple fonction factorielle et/ou produit de
  matrice), nombre et types d'unit�s de calcul, ABI
\item Instructions sp�cialis�es : Synchro m�moire, multim�dia,
  traitement du signal, exemple comment�s d'assembleur
\item Outils logiciels disponibles : compilateurs, librairies, etc
\item Dans quelles machines ces processeurs sont utilisables
\item Autres informations int�ressantes:
  \begin{itemize}
  \item Cas d'usage typique de l'architecture
  \item Exemple d'utilisation
  \end{itemize}
\end{itemize}

\section*{Format de l'examen}

L'objectif est de faire une pr�sentation
\begin{itemize}
\item de 30mn (25 transparents max),
\item par groupe de 4 (si possible 1 th�matique / personne),
\item qui pr�sente de fa�on didactique les informations demand�es.
\item va plus loin que les informations grand public, par exemple pour
  les performances de multiples m�triques sont attendues. 
\end{itemize}

La pr�sentation se fera devant un jury compos� de vos enseignants et
certainement de personnes externes.

Sources d'informations :
\begin{itemize}
\item Wikipedia est un bon d�part
\item Datasheet constructeur
\item Documents marketing
\item Pr�sentations produit pour professionnels
\item ``Applications notes'' des constructeurs
\item Indiquez la liste de vos documents sources
\end{itemize}

Les architectures a �tudier sont a choisir parmi les suivantes :

\begin{itemize}
\item GPU NVIDIA tegra X1 - plus pr�cis�ment sur la partie ARM big.LITTLE
\item GPU NVIDIA tegra X1 - plus pr�cis�ment sur la partie GPU
\item NVIDIA Drive PX - plus pr�cis�ment sur la partie \textit{deep learning}
\end{itemize}

1 groupe par architecture.

\section*{�valuation}

\begin{itemize}
\item La pr�sentation aura lieu le lundi 15 Janvier 2018, � 15h.
\item Vous devrez utiliser des transparents
\item L'�valuation portera sur 
  \begin{itemize}
  \item La qualit� des informations fournies (quantit�, qualit�, pr�cision)
  \item Le niveau d'appropriation de l'architecture
  \item La qualit� didactique de la pr�sentation
  \end{itemize}
\end{itemize}

%\euro{}, \geneuro{}, \geneuronarrow{} et \geneurowide
\end{document}

% Local Variables: ***
% compile-command:"MonMake Document.pdf" ***
% End: ***



