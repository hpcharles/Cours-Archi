\documentclass{article}
\usepackage{eurosym} % \euro{} \geneuro{} \geneuronarrow{} \geneurowide
\usepackage[french]{babel}
% \usepackage{xcolor}
\usepackage{a4}
% \usepackage{url}
\usepackage{sectsty}
\usepackage{hyperref}
\usepackage[utf8]{inputenc}
%\usepackage{helvet}
% \usepackage{graphicx}
% \definecolor{RED}{rgb}{1.,.1,.1}
% \newcommand{\TODO}[1]{\marginpar{\textcolor{RED}{#1}}}
\title{TD Architecture des ordinateurs}
\author{Henri-Pierre Charles \& Frédéric Rousseau}

\begin{document}
\sffamily % http://www.tug.org/pracjourn/2006-1/schmidt/schmidt.pdf
\allsectionsfont{\sffamily}
\maketitle
% \tableofcontents{}
% \TODO{A compléter}
\section{Contexte produit de matrice}

Soit l'algorithme suivant: 
\begin{verbatim}
typedef TYPEELT tMatrix[NLINE][NCOL];

void mulMatrix(tMatrix a, tMatrix b, tMatrix res)
{
  int line, col, k;
  for (line = 0; line < NLINE; line++)
    for (col = 0; col < NCOL; col++)
      {
        res[line][col] = 0;
        for (k = 0; k < NCOL; k++)
          res[line][col] += a[line][k] * b[k][col];
      }
}
\end{verbatim}

Hypothèses : TYPEELT = float; NLINE = NCOL = 1000

Pour un processeur RISC a 1GHz, donner les valeurs théoriques
suivantes :

\begin{enumerate}
\item Combien de MIPS le processeur peut effectuer ?
\item Combien de FLOPS (nombre crête) le processeur peut effectuer ?
\item Combien de temps met le processeur pour effectuer ce calcul ?
\item Combien de FLOPS (nombre crête) le processeur peut effectuer
  en tenant compte des instructions de load store ?
\item Quelle intensité arithmétique atteint on dans ces hypothèses ?
\item Combien de FLOPS (nombre crête) le processeur peut effectuer en
  tenant compte des instructions de load store et une instruction
  ``multiply and add'' ?
\item Quel facteur d'accélération devrait on obtenir ?
\item Combien de FLOPS (nombre crête) le processeur peut effectuer
   en tenant compte des instructions de load store et les effets
   de cache (hypothèse L1 2Ko, pénalité de 100 cycles) ?
\item Même question si l'on fait un produit de matrice 10x10
 \item Combien de temps met le processeur pour effectuer ce calcul
   avec les hypothèses précédentes ?
\end{enumerate}
\newpage{}


\section{Contexte compression d'images}
\begin{verbatim}
#define PIXEL_SAD_C( name, lx, ly ) \
int name( uint8_t *pix1, int i_stride_pix1,  \
                 uint8_t *pix2, int i_stride_pix2 ) \
{                                                   \
    int i_sum = 0;                                  \
    int x, y;                                       \
    for( y = 0; y < ly; y++ )                       \
    {                                               \
        for( x = 0; x < lx; x++ )                   \
        {                                           \
            i_sum += abs( pix1[x] - pix2[x] );      \
        }                                           \
        pix1 += i_stride_pix1;                      \
        pix2 += i_stride_pix2;                      \
    }                                               \
    return i_sum;                                   \
}
\end{verbatim}
CF Algorithme de compression d'image du package VLC. Même processeur
RISC que précédemment
\begin{enumerate}
\item Combien d'instructions pour calculer 1 bloc 16x16 ? (On
  supposera 10 cycles pour \texttt{abs()})
\item Combien de temps pour une image 1920x1080 (1 frame) ? 
\item Peut on atteindre la compression temps réel (25 images /
  secondes)
\item Quel gain peut on atteindre avec l'instruction PSAD/Intel ou
  USADA8/ARM ?
\item Proposez d'autres pistes d'amélioration ?
\end{enumerate}



%\euro{}, \geneuro{}, \geneuronarrow{} et \geneurowide
\end{document}

% Local Variables: ***
% compile-command:"MonMake Document.pdf" ***
% End: ***


