\documentclass{article}
\usepackage{eurosym} % \euro{} \geneuro{} \geneuronarrow{} \geneurowide
% \usepackage[french]{babel}
\usepackage[T1]{fontenc}
\usepackage[utf8]{inputenc}
\usepackage{xcolor}
\usepackage{a4}
% \usepackage{url}
\usepackage{sectsty}
\usepackage{hyperref}
%\usepackage[utf8]{inputenc}
%\usepackage{helvet}
\usepackage{graphicx}
\definecolor{RED}{rgb}{1.,.1,.1}
\newcommand{\TODO}[1]{\marginpar{\textcolor{RED}{#1}}}
\title{TP : Mesure de performances\\sur les mémoires cache (2h)}
\author{Frédéric Rousseau \\Henri-Pierre Charles}
\newcommand{\Image}[2][10]{\includegraphics[width=#1cm]{#2}}

\begin{document}
\sffamily % http://www.tug.org/pracjourn/2006-1/schmidt/schmidt.pdf
\allsectionsfont{\sffamily}
\maketitle
%\tableofcontents{}
%\TODO{A compléter}
%\euro{}, \geneuro{}, \geneuronarrow{} et \geneurowide

L’objectif de cette séance de TP est de montrer l’intérêt des mémoires
cache et leur influence sur les performances d’exécution d’un
programme.

\section{Mesure de temps}
Vous trouverez dans le répertoire \texttt{Partage/TP\_Caches/} un
fichier \texttt{pgm0.c} qui contient un ensemble d’instructions
permettant de mesurer le temps d’exécution d’un programme ou bout de
programme. Dans ce fichier, la fonction \texttt{\_\_rdtscp()} a été
mise en commentaires et remplacer par sa version en langage
d’assemblage. En effet, il existe une instruction \texttt{rdtsc} en
langage d’assemblage x86 qui permet de récupérer le nombre de cycles
du processeur depuis son démarrage
(\href{https://fr.wikipedia.org/wiki/RDTSC}{https://fr.wikipedia.org/wiki/RDTSC}). 

\begin{itemize}
\item Compilez ce programme à l’aide de la commande \texttt{gcc pgm0.c
    -o pgm0}

\item Exécutez-le plusieurs fois. Vous devriez observer quelques
variations mais on mesure quelques cycles d’horloge à 2,93 GHz
(fréquence du processeur des machines de la salle 216 – vous pouvez
vérifier en tapant : \texttt{cat /proc/cpuinfo}).

\item Utilisez une boucle en bash pour l’exécuter une
dizaine de fois. Par exemple avec la commande : 

\texttt{for ((i=0;i<10;i++))do ./pgm0; sleep 1; done}
\end{itemize}

\section{Interêt des lignes de caches}

Le fichier pgm1.c contient la déclaration d’un tableau et son
initialisation. On cherche ensuite à mesurer l’influence des lignes de
caches sur un calcul effectué avec les éléments de ce tableau. 

\begin{itemize}
\item Compilez et exécutez ce programme. 

\item  Dans la bibliothèque \texttt{<x86intrin.h>}, il existe une fonction
\texttt{\_mm\_clflush(\&x);} qui vide la ligne de cache contenant la
variable x. Insérer cette instruction dans la boucle et comparez les
performances en vidant la ligne de cache.

\end{itemize}

\section{Parcours des tableaux}
Le fichier \texttt{pgm2.c} contient la déclaration d’un tableau à 2 dimensions et son initialisation. On cherche ensuite à mesurer l’influence de l’ordre du traitement
lignes-colonne. On souhaite calculer la somme des éléments de ce
tableau.

\begin{itemize}
\item Proposer 2 programmes en inversant l’ordre de traitement
ligne-colonne de ce calcul de la somme des éléments.
\item Indiquez lequel est le plus rapide et pourquoi ? 
\item Avant l’instruction de traitement, videz la ligne de cache et observer la qualité du résultat. 
\item  Vérifier que les résultats restent cohérents pour un tableau de float.
\end{itemize}
\section{Attaque par canal caché}
En janvier 2018, deux failles de sécurité ont été annoncés : Meltdown
et Spectre. Si l’explication technique est complexe, on peut retenir
que le principe est de récupérer des informations présentes dans les
lignes de cache dans certaines conditions. 

Frédéric Pétrot, professeur en informatique à l’ENSIMAG a proposé un
ensemble de tests, de programmes et d’explications pour comprendre
l’idée de ces failles de sécurité :
\url{https://ensiwiki.ensimag.fr/images/5/5b/Sm.pdf} 

Il a par exemple montré qu’il est possible de repérer les lignes de
cache accédées récemment. On appelle cela un canal caché. C’est le
programme pgm3.c légèrement adapté à notre environnement et en
supposant que les lignes des caches sont de 64 octets.

\begin{itemize}
\item Compilez et exécutez le programme. La commande \texttt{gcc
  -DINVALIDATE=0 pgm3.c}. n’exécute pas la boucle d’instructions
  d’invalidation. La commande \texttt{gcc -DINVALIDATE=1 pgm3.c} exécute la
  boucle d’instructions d’invalidation.

\item Observer les résultats sur un ensemble d’exécutions.

\end{itemize}

\section{Transmission d'information par canal caché}

On peut maintenant utiliser cette notion de canal caché pour
transmettre une information entre 2 fonctions. Considérons un
programme principal qui crée un tableau (cf fichier \texttt{side\_channel}
.c). Une fonction side\_channel\_write() invalide une ligne du cache sur
une valeur particulière à transmettre. La fonction side\_channel\_read()
vient ensuite mesurer le temps d’accès aux différents éléments du
tableau. La plus grande valeur correspond alors à la ligne de cache
qui a été invalidée. C’est une méthode pour transmettre une
information d’une fonction à une autre par canal caché.

\begin{itemize}
\item Comprendre le programme \texttt{side\_channel.c}donné. Vous
  noterez par exemple l’utilisation d’un tableau à 2 dimensions (de
  taille 17 pour transmettre une information entre 1 et 16 inclus).
\item Compilez et exécuter le programme en utilisant l’option -O3 de
  gcc.
\item Observer les résultats sur un ensemble d’exécutions.
\end{itemize}

\end{document}

% Local Variables: ***
% compile-command:"MonMake Document.pdf" ***
% End: ***
