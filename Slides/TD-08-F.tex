\pdfminorversion=4
%\documentclass{beamerarticle}
\documentclass{beamer}
\usepackage[T1]{fontenc}
\usepackage[utf8]{inputenc}
\usepackage[french]{babel}
\usepackage{lmodern}
\usepackage{hyperref}
\title{Performance des processeurs multicoeurs}
\subtitle{}
\author{Henri-Pierre Charles \& Frédéric Rousseau}
\date{Année 2016-2017}
\usetheme{Darmstadt}
\newenvironment{Frame}[1]{\subsection{#1}\begin{frame}\frametitle{#1}}{\end{frame}}

\newcommand{\Image}[2][10]{\includegraphics[width=#1cm]{#2}}
\newcommand{\ImageW}[2][2]{\includegraphics[height=#1cm,width=10cm]{#2}}
\newcommand{\Slide}[1]{%
\begin{Frame}{VirtualMemory Synchro}
  \begin{columns}[t]
    \begin{column}{\HW} % Colonne gauche
      \begin{block}{Argumentation}
        \begin{itemize}
        \item 
        \end{itemize}
      \end{block} 
    \end{column}
    
    \begin{column}{\HW} % Colonne droite
      \begin{block}{Illustration}
%        \Image{Synchro/Name.png}
      \end{block}   
    \end{column}
  \end{columns}  
\end{Frame}

%% Local Variables:
%% mode: latex
%% coding: utf-8
%% ispell-dictionary: "american"
%% TeX-master: "../../main.tex"
%% End:

}
\newcommand{\degoal}{\texttt{deGoal\ }}
\graphicspath{{Caches/}{ProgrammingTools/}{VirtualMemory/}{Market/}}

\begin{document}
\begin{frame}
\titlepage
\end{frame}

\begin{frame}
\section{Contexte produit de matrice}

CF L'algo montré dans le cours

Pour un processeur RISC a 1GHz, et un produit de matrice de 1000x1000
en nombres flottants 32 bits, donner les valeurs théoriques suivantes
:

\begin{itemize}
\item Combien de MIPS le processeur peut effectuer ?
\item Combien de FLOPS (nombre crête) le processeur peut effectuer ?
\item Combien de temps met le processeur pour effectuer ce calcul ?
\item Combien de FLOPS (nombre crête) le processeur peut effectuer
  en tenant compte des instructions de load store ?
\item Quelle intensité arithmétique atteint on dans ces hypothèses ?
\item Combien de FLOPS (nombre crête) le processeur peut effectuer en
  tenant compte des instructions de load store et une instruction
  ``multiply and add'' ?
\item Quel facteur d'accélération devrait on obtenir ?
\item Combien de FLOPS (nombre crête) le processeur peut effectuer
   en tenant compte des instructions de load store et les effets
   de cache (hypothèse L1 10Ko, pénalité de 100 cycles) ?
\item Même question si l'on fait un produit de matrice 10x10
 \item Combien de temps met le processeur pour effectuer ce calcul
   avec les hypothèses précédentes ?
\end{itemize}


  
\end{frame}
\end{document}