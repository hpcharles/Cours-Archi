%
\begin{Frame}{Quid de la localité spatiale ?}


      \begin{block}{Profiter de la localité spatiale}
       \begin{center}
 	\begin{itemize}
         \item Dans l'architecture du cache DEC Station 3100, le bloc de données était un mot de 4 octets
         \item Pour profiter de la localité spatiale, il faut que plusieurs donneés consécutives soient dans le même bloc
        \end{itemize}
       \end{center}
      \end{block}   

  

  \begin{block}{Relation adresses mémoire d'un bloc - adresse dans le cache}
    \begin{center}
 	\begin{itemize}
	\item AdresseBlocCache = AdresseBlocMemoire \% NbBlocCache
	\item Simple si la taille du cache est une puissance de 2 (modulo nb de bits)
        \end{itemize}
    \end{center}
  \end{block}   

 

\end{Frame}

%% Local Variables:
%% mode: latex
%% coding: utf-8
%% ispell-dictionary: "american"
%% TeX-master: "../../main.tex"
%% End:

