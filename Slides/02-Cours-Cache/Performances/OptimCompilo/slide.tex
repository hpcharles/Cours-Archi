%


\begin{Frame}{Amélioration des performances par le compilateur}

      \begin{block}{Echange des boucles : améliorer la localité spatiale}
       \begin{center}
 	\begin{itemize}
         \item Hypothèses
		\begin{itemize}
         		\item Les données d'un tableau sont généralement placées à des adresses consécutives en mémoire
			\item Un bloc correspond à plusieurs données
   		\end{itemize}
        \end{itemize}
       \end{center}
      \end{block}   

\vspace{-0,7cm}

  \begin{columns}[t]
    \begin{column}{5cm} % Colonne gauche
      \begin{block}{Avant}
        \begin{lstset}
		afor (j = 0; j < 100; j++) \\
			for (i = 0; i < 5000; i++)  \\
				x[i][j] = 3 * x[i][j];
	\end{lstset}
      \end{block} 
    \end{column}
    
    \begin{column}{5cm} % Colonne droite
      \begin{block}{Après}
	\begin{lstset}
		afor (i = 0; i < 5000; i++) \\
			for (j = 0; j < 100; j++) \\
				x[i][j] = 3 * x[i][j];
	\end{lstset}
	\end{block}
    \end{column}
  \end{columns}  

%\begin{block}{Difficultés}
%       \begin{center}
% 	\begin{itemize}
%         \item 
%        \end{itemize}
%       \end{center}
%      \end{block}   

\begin{alertblock}{Explication}
       \begin{center}
 	\begin{itemize}
         \item Si les valeurs x[a][n], x[a][n+1], x[a][n+2], ... sont à des adresses consécutives, un chargement par blocs permet de placer dans le cache toutes ces valeurs en une seule requête.
        \end{itemize}
       \end{center}
      \end{alertblock}   


\end{Frame}

%% Local Variables:
%% mode: latex
%% coding: utf-8
%% ispell-dictionary: "american"
%% TeX-master: "../../main.tex"
%% End:

