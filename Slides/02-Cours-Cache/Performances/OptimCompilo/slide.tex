%

\begin{frame}[fragile]\frametitle{Amélioration des performances par le programmeur}
  
  \begin{block}{}
    Échange des boucles pour améliorer la localité spatiale. Hypothèses :
    \begin{itemize}
    \item Les données d'un tableau sont généralement placées à des
      adresses consécutives en mémoire
    \item Un bloc correspond à plusieurs données
    \end{itemize}
  \end{block}   
  
  \vspace{-0,7cm}
  
  \begin{columns}[t]
    \begin{column}{6cm} % Colonne gauche
      \begin{block}{}
        \begin{lstlisting}
          for (j = 0; j < 100; j++)// Avant
          for (i = 0; i < 5000; i++)
          x[i][j] = 3 * x[i][j];
\end{lstlisting}
\end{block} 
 \end{column}
    
 \begin{column}{6cm} % Colonne droite
   \begin{block}{}
\begin{lstlisting}
for (i = 0; i < 5000; i++) // Apres
  for (j = 0; j < 100; j++)
    x[i][j] = 3 * x[i][j];
\end{lstlisting}
\end{block}
\end{column}
\end{columns}  

% %\begin{block}{Difficultés}
% %       \begin{center}
% % 	\begin{itemize}
% %         \item 
% %        \end{itemize}
% %       \end{center}
% %      \end{block}   

\begin{alertblock}{Explication}
  Si les valeurs x[a][n], x[a][n+1], x[a][n+2],
  ... sont à des adresses consécutives, un chargement par
  blocs permet de placer dans le cache toutes ces valeurs
  en une seule requête.
\end{alertblock}   
        

\end{frame}

%% Local Variables:
%% mode: latex
%% coding: utf-8
%% ispell-dictionary: "american"
%% TeX-master: "../../main.tex"
%% End:

