%
\begin{Frame}{Que se passe t-il lors d'une écriture ?}


      \begin{block}{2 politiques d'écriture}
       \begin{center}
 	\begin{itemize}
         \item Ecriture immédiate (ou simultannée) (\emph{write though}) : l'information est écrite à la fois dans le cache et dans la mémoire 
         \item Ecriture différée (\emph{write back}) : l'information est écrite uniquement dans le cache. La mémoire ne sera mise à jour que lors d'un remplacement du bloc. 
        \end{itemize}
       \end{center}
      \end{block}   

  


  

  \begin{block}{Avantages}
    \begin{center}
 	\begin{itemize}
        \item L'écriture immédiate est facile à implémenter. Les échecs en lecture ne provoquent jamais d'écriture. Meilleure cohérence de données.
        \item Avec l'écriture différée, plusieurs écritures ne nécessitent qu'une seule écriture en méoire. Solution qui consomme moins.
        \end{itemize}
    \end{center}
  \end{block}   

 

\end{Frame}

%% Local Variables:
%% mode: latex
%% coding: utf-8
%% ispell-dictionary: "american"
%% TeX-master: "../../main.tex"
%% End:

