%
\begin{Frame}{Performances des caches}


      \begin{block}{Critère d'évaluation}
       \begin{center}
 	\begin{itemize}
         \item La pénalité d'échec correspond au temps pour remplacer un bloc dans le cache
         \item Le taux d'échec est la fraction des accès cache qui provoque un échec
        \end{itemize}
       \end{center}
      \end{block}   

  


  

  \begin{block}{Comment améliorer la performance des caches}
    \begin{center}
 	\begin{itemize}
        \item Réduire la pénalité d'échec : cache multi-niveaux, ...
        \item Réduire le temps des accès réussis : cache plus petit, pas de traduction d'adresses, accès pipeliné, ...
	\item \emph{Réduire le taux d'échecs}
        \end{itemize}
    \end{center}
  \end{block}   

 

\end{Frame}

%% Local Variables:
%% mode: latex
%% coding: utf-8
%% ispell-dictionary: "american"
%% TeX-master: "../../main.tex"
%% End:

