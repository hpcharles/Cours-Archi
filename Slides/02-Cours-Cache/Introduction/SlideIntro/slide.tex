
% 
\begin{Frame}{Motivations et principes de base}
  \begin{columns}[t]
    \begin{column}{\BW} % Colonne gauche
      \begin{block}{ Améliorer les temps d'accès à la mémoire}
        Les temps d'accès à la mémoire d'un microprocesseur
        simple ne sont pas très bonnes, en particulier les temps
        d'accès à des mémoires de grande taille. Par exemple, la
        lecture d'une information en RAM peut prendre une centaine de
        cycles processeurs, laissant ce dernier en attente
      \end{block} 
    \end{column}
    
    \begin{column}{\BW} % Colonne droite
      \begin{block}{Idée de principe:}
        \begin{itemize}
        \item Utiliser une mémoire petite mais rapide pour que le processeur
          attende le moins possible
        \end{itemize}
      \end{block}   
    \end{column}
  \end{columns}  

  \begin{block}{Difficultés}
    \begin{center}
      \begin{itemize}
      \item La mémoire cache est une solution générale qui procure en
        \emph{moyenne} un gain de performance significatif. Mais elle
        requiert une partie matérielle spécifique (et logiciel avec un
        compilateur adapté).
        \end{itemize}
       \end{center}
      \end{block}   

\end{Frame}

%% Local Variables:
%% mode: latex
%% coding: utf-8
%% ispell-dictionary: "american"
%% TeX-master: "../../main.tex"
%% End:

