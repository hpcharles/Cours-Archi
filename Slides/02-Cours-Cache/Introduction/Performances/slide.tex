%
\begin{Frame}{Performances}
      \begin{block}{}
        \begin{itemize}
        \item Hypothèses :  1 instruction d'accès à une données nécessite 2 cycles processeur en cas de succès, 10 cycles en cas d'échec.
        \item En fonctionnement normal, on atteint 80\% de succès. 
   	\begin{itemize}
		\item Combien de cycles sont nécessaires pour exécuter 100 accès à une donnée ?
		\item Réponse : (100 x 2 x 0,8) + (100 x 10 x 0,2) = 360 cycles
		\item Combien de cycles le processeur attend ?
		\item Réponse : 100 x (10 - 2) x 0,2 = 160 cycles (environ 44\% du temps)
        \end{itemize}
	\item Les instructions d'accès à une donnée représente 50\% des instructions d'un programme.  
   	\begin{itemize}
		\item Quel est le gain en terme de performance si toutes les données étaient en cache (1 instruction par cycle pour les instructions autres que load et store) ?
	\item Réponse : (50 + 50 x ((2 x 0,8) + (10 x 0,2))) / (50 + 50 x 2)  = 230 / 150 = 1,53
	\end{itemize}
        \end{itemize}
      \end{block} 
      
\end{Frame}

%% Local Variables:
%% mode: latex
%% coding: utf-8
%% ispell-dictionary: "american"
%% TeX-master: "../../main.tex"
%% End:

