\begin{Frame}{Introduction : Rappels Alignement}
 \begin{columns}[t]
  \begin{column}{\BW}
    \begin{block}{Je compte sur mes 16 doigts}\tiny
      \begin{tabular}{|r r r|}\hline
        00000&00 &0                             \\
        00001&01 &1                             \\
        00010&02 &2                             \\
        00011&03 &3                             \\ \hline
        00100&04 &4                             \\
        00101&05 &5                             \\
        00110&06 &6                             \\
        00111&07 &7                             \\ \hline \hline
        01000&08 &8                             \\
        01001&09 &9                             \\
        01010&10&A                             \\
        01011&11&B                             \\ \hline
        01100&12&C                             \\
        01101&13&D                             \\
        01110&14&E                             \\
        01111&15&F                             \\ \hline
      \end{tabular}
    \end{block}
  \end{column}
  \begin{column}{\BW}
    \begin{block}{Caractéristiques d'une addresse}
      \begin{itemize}
      \item Adresses = adresses d'octets
      \item Adresses de mots de 32 bits = multiples de 4 (2
        bits à 0)
      \item Adresses de mots de 64 bits = multiples de 8 (3
        bits à 0)
      \item etc
      \end{itemize}
    \end{block}
    \begin{block}{}
      \begin{itemize}
      \item  Modulo = garder la partie droite d'un mot binaire
      \item  Diviser = garder la partie gauche d'un mot binaire
      \end{itemize}
    \end{block}
  \end{column}

 \end{columns}
\end{Frame}
%% Local Variables:
%% mode: latex
%% mode: flyspell
%% coding: utf-8
%% ispell-dictionary: "american"
%% TeX-master: "../../main.tex"
%% End:
