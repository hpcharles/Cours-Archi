\begin{Frame}{Introduction : Rappels Architecture 64 bits}
    Que veut dire ``Architecture 64 bits'' ? (Rien, il faut préciser)
 \begin{columns}[t]
  \begin{column}{\BW}

    \begin{block}{3 possibilités}
      \begin{itemize}
      \item Largeur du bus adresse : les adresses 32 bits limitent le
        nombre d'adresse possibles à $2^{32}= 4 294 967 296$ c'est à
        dire 4 giga octets.
        \item Largeur du bus données : cela permet de récupérer 64
          bits en une seule transaction mémoire
        \item Largeur des données manipulées dans les ALU : utiliser
          des flottants double ou des entiers longs
      \end{itemize}
    \end{block}
  \end{column}
  \begin{column}{\BW}
    \begin{block}{Conséquences}
      \begin{itemize}
      \item En pratique, les bus adresses sont de 41 bits (pas 64)
      \item Moins de transaction mémoire pour traiter des données
        vectorielles
      \item Traiter des données de grande précision
      \end{itemize}
    \end{block}
  \end{column}

 \end{columns}
\end{Frame}
%% Local Variables:
%% mode: latex
%% mode: flyspell
%% coding: utf-8
%% ispell-dictionary: "american"
%% TeX-master: "../../main.tex"
%% End:
