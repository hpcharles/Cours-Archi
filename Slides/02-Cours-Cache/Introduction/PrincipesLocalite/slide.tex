%
\begin{Frame}{Principe de localité}

  \begin{block}{Localité temporelle}
    \begin{center}
 	\begin{itemize}
        \item Les données ou instructions utilisées récemment le seront encore dans un futur proche (boucles, tableaux)
        \item Dans un programme, 90\% du temps est passé dans seulement 10\% des instructions
        \end{itemize}
    \end{center}
  \end{block}   

  \begin{block}{Localité spatiale}
    \begin{center}
 	\begin{itemize}
        \item Les données ou instructions proches en mémoire seront utilisées en même temps (tableaux, instructions séquentielles)
        \end{itemize}
    \end{center}
  \end{block}   

  \begin{alertblock}{Par conséquent}
    \begin{center}
 	\begin{itemize}
        \item Placer les données et les instructions en cours d'utilisation proche du processeur
        \item Les données ou instructions accédées le plus souvent doivent être proche du processeur
        \end{itemize}
    \end{center}
  \end{alertblock}   

\end{Frame}

%% Local Variables:
%% mode: latex
%% coding: utf-8
%% ispell-dictionary: "american"
%% TeX-master: "../../main.tex"
%% End:

