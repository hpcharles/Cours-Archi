\documentclass{article}
\usepackage{eurosym} % \euro{} \geneuro{} \geneuronarrow{} \geneurowide
\usepackage[french]{babel}
% \usepackage{xcolor}
\usepackage{a4}
% \usepackage{url}
\usepackage{sectsty}
\usepackage{hyperref}
\usepackage{xcolor}
\usepackage{listings}
\lstset{ %
  basicstyle=\footnotesize,        % the size of the fonts that are used for the code
  breakatwhitespace=false,         % sets if automatic breaks should only happen at whitespace
  breaklines=true,                 % sets automatic line breaking
  commentstyle=\color{green},    % comment style
  extendedchars=true,              % lets you use non-ASCII characters; for 8-bits encodings only, does not work with UTF-8
  frame=single,	                   % adds a frame around the code
  keepspaces=true,                 % keeps spaces in text, useful for keeping indentation of code (possibly needs columns=flexible)
  keywordstyle=\color{blue},       % keyword style
  language=C,                 % the language of the code
  numbers=right,                    % where to put the line-numbers; possible values are (none, left, right)
  numbersep=5pt,                   % how far the line-numbers are from the code
  numberstyle=\tiny\color{blue}, % the style that is used for the line-numbers
  rulecolor=\color{black},         % if not set, the frame-color may be changed on line-breaks within not-black text (e.g. comments (green here))
  showspaces=false,                % show spaces everywhere adding particular underscores; it overrides 'showstringspaces'
  stringstyle=\color{mymauve},     % string literal style
  tabsize=2,	                   % sets default tabsize to 2 spaces
}

\usepackage[utf8]{inputenc}
%\usepackage{helvet}
% \usepackage{graphicx}
% \definecolor{RED}{rgb}{1.,.1,.1}
% \newcommand{\TODO}[1]{\marginpar{\textcolor{RED}{#1}}}
\title{TD Architecture des ordinateurs}
\author{Henri-Pierre Charles \& Frédéric Rousseau}

\begin{document}
\sffamily % http://www.tug.org/pracjourn/2006-1/schmidt/schmidt.pdf
\allsectionsfont{\sffamily}
\maketitle
% \tableofcontents{}
% \TODO{A compléter}
\section{Contexte produit de matrice}

Soit l'algorithme suivant: 
\lstinputlisting{MatrixKernel.c}

Hypothèses :
\begin{itemize}
\item TYPEELT = float (32 bits); 
\item NLINE = NCOL = 1000
\end{itemize}

Pour un processeur RISC a 1GHz, qui peut effectuer une multiplication
flottante en 1 cycle, donner les valeurs théoriques suivantes :

\begin{enumerate}
\item Combien de MIPS le processeur peut effectuer ?
  
  % 1000 MIPS
  
\item Combien de FLOP (nombre crête) le processeur peut effectuer,
  s'il n'y a pas de Load / Store ?
  
  % 1000 MFLOPS = 1GFLOPS
  
\item Combien de temps met le processeur pour effectuer le calcul du
  produit de matrice (en suppossant que la mémoire n'est pas un
  facteur de ralentissement et que les instructions de la boucle
  n'interviennent pas) ?
  
  % 2 FLOPS / itération + et *
  % NLINE * NCOL * 2 opérations = 2M FLOPS
  % 2 secondes 
  
\item Combien de FLOPS (nombre crête) le processeur peut effectuer (en
  tenant compte des instructions de load store) ?
  
  % Kernel = Ld Ld Float Float Store (2 FLOP pour 5 cycles)
  % 0.25 GFLOPS
  
\item Quelle intensité arithmétique atteint on dans ces hypothèses ?
  
  % Flop / Byte 
\item Combien de FLOPS (nombre crête) le processeur peut effectuer en
  tenant compte des instructions de load store et une instruction
  ``multiply and add'' ?
\item Quel facteur d'accélération devrait on obtenir ?
\item Combien de FLOPS (nombre crête) le processeur peut effectuer
  en tenant compte des instructions de load store et les effets
  de cache (hypothèse L1 2Ko, pénalité de 1000 cycles) ?
\item Même question si l'on fait un produit de matrice 10x10
\item Combien de temps met le processeur pour effectuer ce calcul
  avec les hypothèses précédentes ?
\item Que peut on faire de mieux pour accélérer le code ?
  \begin{itemize}
  \item Que peut on changer sur la ligne \texttt{ res[line][col] = 0;} ?
  \item Comment peut on mieux optimiser pour les caches ?
  \item Comment paralléliser ?
  \end{itemize}
\item Si un générateur de code dynamique met 20 M cycles pour générer
  une instruction et que le kernel fait 40 instructions, mais permet
  d'accélérer d'un facteur 2, pour quelle taille de matrice a-t-on un
  gain ? Même question pour si 200 cycles / instruction.
\end{enumerate}

\section{Contexte compression d'images}

\lstinputlisting{PSAD.c}


CF Algorithme de compression d'image du package VLC. Même processeur
RISC que précédemment
\begin{enumerate}
\item Combien d'instructions pour calculer 1 bloc 16x16 ? (On
  supposera 10 cycles pour \texttt{abs()})
\item Combien de temps pour une image 1920x1080 (1 frame) ? 
\item Peut on atteindre la compression temps réel (25 images /
  secondes)
\item Quel gain peut on atteindre avec l'instruction PSAD/Intel ou
  USADA8/ARM ?
\item Proposez d'autres pistes d'amélioration ?
\end{enumerate}


\end{document}

% Local Variables: ***
% compile-command:"MonMake Document.pdf" ***
% End: ***


