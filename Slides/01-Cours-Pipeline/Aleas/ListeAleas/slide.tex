%
\begin{Frame}{Aléas de pipeline}

\begin{block}{On ne peut pas toujours exécuter l'instruction suivante au cycle d'horloge suivant :}
       \begin{center}
 	\begin{itemize}

         \item Aléas de données (dépendances de données)
		\begin {itemize}
         		\item Instruction ayant besoin du résultat de l'instruction précédente
		\end{itemize}

	 \item Aléas de contrôle (branchements)
		\begin {itemize}
         		\item La prochaine instruction à exécuter n'est connue qu'à la fin du traitement - condition de branchement
		\end{itemize}

	 \item Aléas de contrôle (interruption ou exception)
		\begin {itemize}
         		\item Les interruptions ou exceptions viennent modifier le cours de l'exécution du programme.
		\end{itemize}

	 \item Aléas structurels (conflits de ressources)
		\begin {itemize}
         		\item Accès au même composant au même cycle (lecture-écriture dans le même registre par exemple)
		\end{itemize}

        \end{itemize}
       \end{center}
      \end{block}   

 

%\begin{block}{Solutions}
%       \begin{center}
% 	\begin{itemize}
%         \item Gel de pipeline
%	 \item Astuces d'architecture : registre double port, transfert décalé
%	 \item Prédiction de branchements
%        \end{itemize}
%       \end{center}
%      \end{block}   

\end{Frame}

%% Local Variables:
%% mode: latex
%% coding: utf-8
%% ispell-dictionary: "american"
%% TeX-master: "../../main.tex"
%% End:

