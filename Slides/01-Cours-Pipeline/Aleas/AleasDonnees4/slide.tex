%
\begin{Frame}{Aléas de donnée - Les solutions matérielles}

  \begin{block}{L'envoi}
       \begin{center}
 	\begin{itemize}
          \item On constate sur l'exemple ci-dessous que la donnée \$2 est disponible à la sortie de EXE, et utilisée à l'entrée de EXE de l'instruction suivante.
          \item On peut utiliser les résultats temporaires (disponibles dans les registres du pipeline) en entrée de l'ALU (EXE). Le pipeline peut alors s'exécuter sans suspension. 
        \end{itemize}
       \end{center}
      \end{block}   

\vspace{-0.8cm}
        \begin{center}
           \Image[12]{/Graphiques/AleaDonnee4.pdf}
        \end{center}
 

%\begin{block}{Solutions}
%       \begin{center}
% 	\begin{itemize}
%         \item Gel de pipeline
%	 \item Astuces d'architecture : registre double port, transfert décalé
%	 \item Prédiction de branchements
%        \end{itemize}
%       \end{center}
%      \end{block}   

\end{Frame}

%% Local Variables:
%% mode: latex
%% coding: utf-8
%% ispell-dictionary: "american"
%% TeX-master: "../../main.tex"
%% End:

