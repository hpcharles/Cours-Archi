%
\begin{Frame}{Exemple de code g\'en\'er\'e par gcc (pour le processeur MIPS)}

 
	\begin{block} {A quoi servent les NOP ?}

\$L2:\\
\hspace{0,5cm} . . .\\
\hspace{0,5cm} lw \hspace{0,5cm}\$2,32(\$fp) \\
\hspace{0,5cm} lw \hspace{0,5cm}\$3,36(\$fp)\\
\hspace{0,5cm} \textcolor{tagada}{nop} 	\hspace{0,99cm}\# nop5\\
\hspace{0,5cm} slt   \hspace{0,5cm}\$2,\$2,\$3\\
\hspace{0,5cm} beq \hspace{0,35cm}\$2,\$0,\$L5\\
\hspace{0,5cm} \textcolor{tagada}{nop}		\hspace{0,99cm}\# nop6\\
\hspace{0,5cm} li    \hspace{0,7cm}\$2,-1\\
\hspace{0,5cm} sw   \hspace{0,5cm}\$2,16(\$fp)\\
\hspace{0,5cm} j      \hspace{0,8cm}\$L4\\
\hspace{0,5cm} \textcolor{tagada}{nop}	\hspace{0,99cm}\# nop7\\
\hspace{0,5cm} . . .\\

\end{block}


      
\end{Frame}

%% Local Variables:
%% mode: latex
%% coding: utf-8
%% ispell-dictionary: "american"
%% TeX-master: "../../main.tex"
%% End:

