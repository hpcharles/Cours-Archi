%
\begin{Frame}{Aléas de contrôle - Interruptions et exceptions}

  \begin{columns}[t]
    \begin{column}{\BW} % Colonne gauche
      \begin{block}{Interruption par un périphérique}
        \begin{center}
          \begin{itemize}
          \item La solution la plus simple est de terminer l'exécution des
            instructions qui sont dans le pipeline avant de traiter
            l'interruption.
          \end{itemize}
        \end{center}
      \end{block}
      \begin{block}{Attention : comportement dépéndant de l'architecture}
        \Image{AleasITExc1/DoesIntGenerateInterrupt}
      \end{block}
    \end{column}
    \begin{column}{\BW} % Colonne droite
      \begin{block}{Exception - div x15, x16, x16}
        \begin{center}
          \begin{itemize}
          \item Supposons que l'instruction div provoque un débordement
            qui provoque une exception
          \item Il ne faut pas contaminer les registres des instructions
            suivantes qui sont dans le pipeline. Il faut vider le pipeline
            avec un méchanisme identique à celui des aléas de
            branchements. On peut ensuite exécuter la routine d'exception.
          \end{itemize}
        \end{center}
      \end{block}   
    \end{column}  
  \end{columns}  
\end{Frame}

%% Local Variables:
%% mode: latex
%% coding: utf-8
%% ispell-dictionary: "american"
%% TeX-master: "../../main.tex"
%% End:

