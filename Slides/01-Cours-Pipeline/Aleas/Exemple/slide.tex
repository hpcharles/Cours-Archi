%
\begin{Frame}{Un exemple sur le MIPS - solutions pour les aléas}

\begin{block}{On considère l'architecture du MIPS pipeliné}
       \begin{center}
 	\begin{itemize}
	\item add \$t1, \$t2, \$t3  \hspace*{0,3cm} IF \hspace*{0,3cm} DE \hspace*{0,3cm} EXE \hspace*{0,3cm} MEM \hspace*{0,3cm} WB
	\item add \$t4, \$t5, \$t1  \hspace*{1,2cm} IF \hspace*{0,45cm} DE \hspace*{0,5cm}  EXE      \hspace*{0,4cm} MEM \hspace*{0,4cm} WB
%	\begin{tabular}{llllllll}
%         {add \$t1, \$t2, \$t3}&&IF&DE&EXE&MEM&WB&\\
%	 {add \$t4, \$t5, \$t1}&&&IF&DE&EXE&MEM&WB
%	\end{tabular}
	\item Matériel : bypass entre la sortie de l'étage EXE et son entrée
	\item Logiciel : 3 NOP
        \end{itemize}
       \end{center}
      \end{block}   
 

\vspace{-0.1cm}
        \begin{center}
           \Image[6]{/Graphiques/Archi-MIPS3.pdf}
        \end{center}
      
\end{Frame}

%% Local Variables:
%% mode: latex
%% coding: utf-8
%% ispell-dictionary: "american"
%% TeX-master: "../../main.tex"
%% End:

