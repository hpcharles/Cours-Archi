%
\begin{Frame}{Idée de fonctionnement}
%  \begin{columns}[t]
%    \begin{column}{7cm} % Colonne gauche
%      \begin{block}{Motivation}
%        \begin{itemize}
%        \item Les performances d'un microprocesseur simple ne sont pas très bonnes
%	\item Ce n'est pas une technique nouvelle : les premières machines commerciales datent de 1964 avec l'IBM 360/91
%        \end{itemize}
%      \end{block} 
%    \end{column}
%    
%    \begin{column}{4cm} % Colonne droite
%      \begin{block}{idée}
%%        \Image{SlideIntro/Name.png}
%\begin{itemize}
%        \item Le pipeline est une technique de réalisation dans laquelle plusieurs instructions se supersposent pendant l'exécution
%        \end{itemize}
%      \end{block}   
%    \end{column}
%  \end{columns}  

\begin{block}{Principe appliqué aux processeurs}
       \begin{center}
 	\begin{itemize}
         \item Les activités au sein de l'unité centrale sont organisées comme une chaîne de montage : on décompose l'unité de commande chargée de la \textbf{lecture de instructions}, de leur \textbf{décodage}, de leur \textbf{exécution}, ... en plusieurs modules fonctionnels.
	\item Ces \textbf{modules travaillent en parallèle}
	\item On commence ainsi le traitement d'une nouvelle instruction avant que la précédente ne soit terminée, ce qui crée un flot continu en entrée de chacun des modules.
        \end{itemize}
       \end{center}
      \end{block}   



\end{Frame}

%% Local Variables:
%% mode: latex
%% coding: utf-8
%% ispell-dictionary: "american"
%% TeX-master: "../../main.tex"
%% End:

