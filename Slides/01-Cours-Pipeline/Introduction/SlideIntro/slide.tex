%
\begin{Frame}{Motivations et principes de base}
  \begin{columns}[t]
    \begin{column}{7cm} % Colonne gauche
      \begin{block}{Motivation}
        \begin{itemize}
        \item Les performances d'un microprocesseur simple ne sont pas
          très bonnes. Pourquoi ?
        \item Technique qui n'est pas nouvelle : les premières
          machines commerciales datent de 1964 avec l'IBM 360/91
        \end{itemize}
      \end{block} 
    \end{column}
    
    \begin{column}{4,5cm} % Colonne droite
      \begin{block}{Idée}
%        \Image{SlideIntro/Name.png}
        \begin{itemize}
        \item Le pipeline est une technique de réalisation dans
          laquelle plusieurs instructions se supersposent pendant
          l'exécution
        \end{itemize}
      \end{block}   
    \end{column}
  \end{columns}  

  \begin{block}{Difficultés}
       \begin{center}
 	\begin{itemize}
    \item Ce recouvrement est source de nombreuses difficultés dans
      l'exécution des instructions
        \end{itemize}
       \end{center}
      \end{block}   

\end{Frame}

%% Local Variables:
%% mode: latex
%% coding: utf-8
%% ispell-dictionary: "american"
%% TeX-master: "../../main.tex"
%% End:

