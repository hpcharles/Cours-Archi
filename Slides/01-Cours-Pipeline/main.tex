\pdfminorversion=4 % -*- latex-mode -*-
\documentclass[aspectratio=169]{beamer}
\usepackage[T1]{fontenc}
\usepackage[utf8]{inputenc}
\usepackage{lmodern}
\usepackage{multimedia}
\usepackage{listings}
\lstset{language=C,numbers=left,stepnumber=1,basicstyle=\ttfamily\scriptsize}
\newcommand*\lstinputpath[1]{\lstset{inputpath=#1}}

\definecolor{tagada}{RGB}{250,50,50}
\setbeamertemplate{navigation symbols}{\relax}

\title{Le PIPELINE d'exécution des instructions d'un processeur}
\author{Henri-Pierre Charles \& Frédéric Rousseau}
\date{}
\usetheme{Darmstadt}
\setbeamercovered{dynamic}
%\newenvironment{Frame}[1]{\subsection{#1}\begin{frame}\frametitle{#1}}{\end{frame}}
\newenvironment{Frame}[1]{\begin{frame}\frametitle{#1}}{\end{frame}}
\newcommand{\W}[2][en]{\href{https://#1.wikipedia.org/wiki/#2}{\detokenize{#2}}}
\newcommand{\Image}[2][11cm]{\includegraphics[width=#1]{#2}}
\newcommand{\ImageW}[2][2]{\includegraphics[height=#1,width=10cm]{#2}}
\newcommand{\HREF}[2][https]{\href{#1://#2}{#1://#2}}
\newcommand{\Slide}[1]{%
\begin{Frame}{VirtualMemory Synchro}
  \begin{columns}[t]
    \begin{column}{\HW} % Colonne gauche
      \begin{block}{Argumentation}
        \begin{itemize}
        \item 
        \end{itemize}
      \end{block} 
    \end{column}
    
    \begin{column}{\HW} % Colonne droite
      \begin{block}{Illustration}
%        \Image{Synchro/Name.png}
      \end{block}   
    \end{column}
  \end{columns}  
\end{Frame}

%% Local Variables:
%% mode: latex
%% coding: utf-8
%% ispell-dictionary: "american"
%% TeX-master: "../../main.tex"
%% End:

}
\graphicspath{{Introduction/}{GainProcesseurs/}{PipelineRISCV/}{Aleas/}{Conclusion/}}

\begin{document}
\begin{frame}
\titlepage
\end{frame}


\begin{frame}
\tableofcontents[
%currentsubsection,
%hideothersubsections,
%sectionstyle=show/hide,
%subsectionstyle=show/shaded,
]
\end{frame}


\section{Introduction}
\Slide{Introduction/SlideIntro}
\Slide{Introduction/PrincipeAlgoProcesseur}
\Slide{Introduction/Principe}
\Slide{Introduction/AnalogieSansPipeline}
\Slide{Introduction/AnalogieAvecPipeline}
\Slide{Introduction/PerfLaverie}

\section{GainProcesseurs}
\Slide{GainProcesseurs/Interet1}
\Slide{GainProcesseurs/Interet2}
\Slide{GainProcesseurs/Interet3}

\section{PipelineRISCV}
\Slide{PipelineRISCV/ArchiRISCV1}
\Slide{PipelineRISCV/ArchiRISCV2}
\Slide{PipelineRISCV/ArchiRISCV3}
\Slide{PipelineRISCV/ArchiRISCV4}
\Slide{PipelineRISCV/ArchiRISCV5}
\Slide{PipelineRISCV/ArchiRISCV6}
\Slide{PipelineRISCV/ArchiRISCV7}
\Slide{PipelineRISCV/ArchiRISCV8}

\section{Aleas}
\Slide{Aleas/PrincipeAleas}
\Slide{Aleas/ListeAleas}
\Slide{Aleas/AleasDonnees1}
\Slide{Aleas/AleasDonnees2}
\Slide{Aleas/AleasDonnees3}
\Slide{Aleas/AleasDonnees4}
\Slide{Aleas/AleasDonnees5}

\Slide{Aleas/AleasBranch1}
\Slide{Aleas/AleasBranch2}
\Slide{Aleas/AleasBranch3}
\Slide{Aleas/ExempleCodeNop}
\Slide{Aleas/AleasITExc1}

\Slide{Conclusion/Conclusion}
\Slide{Conclusion/ExerciceTD}

\end{document}



% Local Variables: ***
% compile-command:"MonMake main.pdf" ***
% coding: utf-8
% End: ***
