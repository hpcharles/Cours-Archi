\pdfminorversion=4
\documentclass{beamer}
\usepackage[T1]{fontenc}
\usepackage[utf8]{inputenc}
\usepackage[french]{babel}
\usepackage{lmodern}
\usepackage{multimedia}
\usepackage{listings}
\lstset{language=C,numbers=left,stepnumber=1,basicstyle=\ttfamily\scriptsize}
\newcommand*\lstinputpath[1]{\lstset{inputpath=#1}}
 
\title{Le PIPELINE}
\author{Henri-Pierre Charles \& Frédéric Rousseau}
\date{}
\usetheme{Darmstadt}
\setbeamercovered{dynamic}
\newenvironment{Frame}[1]{\subsection{#1}\begin{frame}\frametitle{#1}}{\end{frame}}

\newcommand{\Image}[2][11]{\includegraphics[width=#1cm]{#2}}
\newcommand{\ImageW}[2][2]{\includegraphics[height=#1cm,width=10cm]{#2}}
\newcommand{\Slide}[1]{%
\begin{Frame}{Que se passe t-il lors d'une écriture ?}


  \begin{block}{2 politiques d'écriture}
    \begin{center}
      \begin{itemize}
      \item Ecriture immédiate (ou simultannée) (\emph{write though}) :
        l'information est écrite à la fois dans le cache et dans la
        mémoire
      \item Ecriture différée (\emph{write back}) : l'information est
        écrite uniquement dans le cache. La mémoire ne sera mise à jour
        que lors d'un remplacement du bloc.
      \end{itemize}
    \end{center}
  \end{block}   

  \begin{block}{Avantages}
    \begin{center}
 	\begin{itemize}
    \item L'écriture immédiate est facile à implémenter. Les échecs en
      lecture ne provoquent jamais d'écriture. Meilleure cohérence de
      données.
        \item Avec l'écriture différée, plusieurs écritures ne
          nécessitent qu'une seule écriture en mémoire. Solution qui
          consomme moins.
        \end{itemize}
    \end{center}
  \end{block}   


\end{Frame}

%% Local Variables:
%% mode: latex
%% coding: utf-8
%% ispell-dictionary: "american"
%% TeX-master: "../../main.tex"
%% End:

}
\graphicspath{{Introduction/}{Partie1}}


\begin{document}
\begin{frame}
\titlepage
\end{frame}


\begin{frame}
\tableofcontents[ 
currentsubsection, 
hideothersubsections, 
sectionstyle=show/hide, 
subsectionstyle=show/shaded, 
] 
\end{frame}


\section{Introduction}
\Slide{Introduction/SlideIntro}
\Slide{Introduction/Principe}
\Slide{Introduction/AnalogieSansPipeline}
\Slide{Introduction/AnalogieAvecPipeline}

\section{Performances}
\Slide{Performances/Interet1}
\Slide{Performances/interet2}
\Slide{Performances/Interet3}

\section{PipelineMIPS}
\Slide{PipelineMIPS/ArchiMIPS1}
\Slide{PipelineMIPS/ArchiMIPS2}
\Slide{PipelineMIPS/ArchiMIPS3}

\section{Aleas}
\Slide{Aleas/PrincipeAleas}
\Slide{Aleas/ListeAleas}
\Slide{Aleas/Exemple}

%\section{Introduction}
%\Slide{Introduction/SlideIntro}
%\Slide{Introduction/Principe}
%\Slide{Introduction/AnalogieSansPipeline}
%\Slide{Introduction/AnalogieAvecPipeline}
%
%\section{CorrespondanceDirecte}
%\Slide{CorrespondanceDirecte/Principes}
%\Slide{CorrespondanceDirecte/ArchiGenerale}
%\Slide{CorrespondanceDirecte/Questions1}
%\Slide{CorrespondanceDirecte/DECStation_3100}
%\Slide{CorrespondanceDirecte/Principes2}
%\Slide{CorrespondanceDirecte/ArchiGeneraleBloc}
%
%\section{Associatif}
%\Slide{Associatif/Principes}
%\Slide{Associatif/ALPHA21264}
%
%\section{Performances}
%\Slide{Performances/Principes}


\end{document}



% Local Variables: ***
% compile-command:"MonMake main.pdf" ***
% coding: utf-8
% End: ***
