%
\begin{Frame}{Intérêt - Application aux microprocesseurs}


\begin{block}{Architecture sans pipeline - 1 cycle = 1 ns pour chaque étage}
       \begin{center}
 	\begin{itemize}
         \item Pour exécuter 10 instructions, il faut 10 cycles soit 10 ns
	\item Pour éxécuter \emph{m} instructions, il faut \emph{m} cycles
        \end{itemize}
       \end{center}
      \end{block}   

\begin{block}{Architecture avec pipeline - 5 étages}
       \begin{center}
 	\begin{itemize}
	 \item 1 instruction en 1 ns, soit 0,2 ns si tous les étages ont le même temps de traitement
         \item Pour exécuter 10 instructions, il faut 14 étapes soit 2,8 ns
%	 \item Pour exécuter 1000 instructions, il faut 1004 étapes soit 200,8 ns \approx{1000/5}
        \end{itemize}
       \end{center}
      \end{block}   

\begin{block}{Généralisation pour une architecture pipeline à \emph{n} étages}
       \begin{center}
 	\begin{itemize}
         \item Si m le nombre d'instructions est très grand, le temps d'exécution est
	$ \frac{m+n-1}{n} = \frac{m}{n} $
        \end{itemize}
       \end{center}
      \end{block}   

\end{Frame}

%% Local Variables:
%% mode: latex
%% coding: utf-8
%% ispell-dictionary: "american"
%% TeX-master: "../../main.tex"
%% End:

