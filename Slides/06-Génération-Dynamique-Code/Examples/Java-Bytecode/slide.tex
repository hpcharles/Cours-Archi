%
\begin{Frame}{Example : Java-Bytecode}
  \begin{columns}[t]
    \begin{column}{\BW} % Colonne gauche
      \begin{block}{Java context usages}
        \begin{itemize}
        \item Web browser : speed and reactivity
        \item Server : reasonnable speed quickly
        \item HPC : optimallity slowly
        \end{itemize}
      \end{block} 
      \begin{block}{Optimization level}
        \begin{itemize}
        \item Interpretor : slow but reactive
        \item C1 : JIT compiler
        \item C2 : Complex JIT compiler, very slow
        \end{itemize}
      \end{block} 
    \end{column}
    
    \begin{column}{\BW} % Colonne droite
      \begin{block}{Compiler transition}
        \Image[5]{Java-Bytecode/Staged.pdf}
      \end{block}
      Green : Staged mode; Red : normal mode; Green : backtrack
    \end{column}
  \end{columns}  
\end{Frame}

%% Local Variables:
%% mode: latex
%% coding: utf-8
%% ispell-dictionary: "american"
%% TeX-master: "../../main.tex"
%% End:

