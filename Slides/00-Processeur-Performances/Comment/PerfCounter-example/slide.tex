\begin{frame}[fragile]{Comment : Perf counter example}

  \begin{columns}[t]
   \begin{column}{\BW}
     \begin{block}{Code Example}
\lstinputlisting{Comment/PerfCounter-example/PerfCounter.c}
     \end{block}
   \end{column}
   \begin{column}{\BW}
     \begin{block}{Steps}
       \begin{itemize}
       \item Instrument code
       \item use PerfCounter
       \item Print results
       \end{itemize}
     \end{block}
     \begin{block}{Results}
       \begin{itemize}
       \item Cycle ?
       \item Additions / Multiplications ?
       \item ../..
       \end{itemize}
     \end{block}
   \end{column}
  \end{columns}
\end{frame}
%
\begin{Frame}{How : }
  \begin{columns}[t]
    \begin{column}{\BW} % Colonne gauche
      \begin{block}{Performance counter from HW}
        \begin{itemize}
        \item Simple one : \texttt{cycle counter}
        \item More complex : \texttt{RAT\_STALLS} Counts the number of cycles during which execution stalled due to several reason
        \item L2, L3 cache access, hit, miss ...
        \item Miss prediction, etc
        \end{itemize}
      \end{block} 
      \begin{alertblock}{Question}
        \begin{itemize}
        \item Comment calcule-t-on l'IPC ?
        \end{itemize}
      \end{alertblock}
    \end{column}
    
    \begin{column}{\BW} % Colonne droite
      \begin{block}{Tools}
        \begin{itemize}
        \item Intel tools : \url{http://www.intel.com/software/pcm}
        \item Papi : \url{http://icl.cs.utk.edu/papi}
        \item Tiptop from Inria : \url{http://tiptop.gforge.inria.fr/}
        \end{itemize}
%        \Image{PerfCounter/Name.png}
      \end{block}   
      \begin{alertblock}{Limitations}
        \begin{itemize}
        \item Complicated to understand / analyse
        \item No real portable library
        \end{itemize}
      \end{alertblock}   
    \end{column}
  \end{columns}  
\url{https://en.wikipedia.org/wiki/Hardware_performance_counter}
\end{Frame}

%% Local Variables:
%% mode: latex
%% coding: utf-8
%% ispell-dictionary: "american"
%% TeX-master: "../../main.tex"
%% End:

