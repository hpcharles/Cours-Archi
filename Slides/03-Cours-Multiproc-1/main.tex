\pdfminorversion=4
%\documentclass{beamerarticle}
\documentclass[aspectratio=169]{beamer}
\usepackage[T1]{fontenc}
\usepackage[utf8]{inputenc}
% \usepackage[french]{babel}
\usepackage{lmodern}
\usepackage{multicol}
\usepackage{hyperref}
\usepackage{algorithm2e}
\usepackage{listings}
 \lstset{ 
  basicstyle=\tiny,        % the size of the fonts that are used for the code
  keepspaces=true,                 % keeps spaces in text, useful for keeping indentation of code (possibly needs columns=flexible)
  keywordstyle=\color{blue},       % keyword style
  commentstyle=\color{red},
  language=C,                 % the language of the code
  numbers=left,                    % where to put the line-numbers; possible values are (none, left, right)
  numbersep=5pt,                   % how far the line-numbers are from the code
  showtabs=false,                  % show tabs within strings adding particular underscores
  stepnumber=2,                    % the step between two line-numbers. If it's 1, each line will be numbered
  tabsize=2,	                   % sets default tabsize to 2 spaces
}

\title{Performance des processeurs multic{\oe}urs}
\subtitle{}
\author{Henri-Pierre Charles \& Frédéric Rousseau}
\date{}
\usetheme{Darmstadt}
\newenvironment{Frame}[1]{\subsection{#1}\begin{frame}\frametitle{#1}}{\end{frame}}

\newcommand{\Image}[2][10]{\includegraphics[width=#1cm]{#2}}
\newcommand{\ImageW}[2][2]{\includegraphics[height=#1cm,width=10cm]{#2}}
\newcommand{\Slide}[1]{%
\begin{Frame}{VirtualMemory Synchro}
  \begin{columns}[t]
    \begin{column}{\HW} % Colonne gauche
      \begin{block}{Argumentation}
        \begin{itemize}
        \item 
        \end{itemize}
      \end{block} 
    \end{column}
    
    \begin{column}{\HW} % Colonne droite
      \begin{block}{Illustration}
%        \Image{Synchro/Name.png}
      \end{block}   
    \end{column}
  \end{columns}  
\end{Frame}

%% Local Variables:
%% mode: latex
%% coding: utf-8
%% ispell-dictionary: "american"
%% TeX-master: "../../main.tex"
%% End:

}
\newcommand{\degoal}{\texttt{deGoal\ }}
\newcommand{\HW}{.5\textwidth}
\newcommand{\WP}[1]{\href{https://en.wikipedia.org/wiki/#1}{#1}}

\graphicspath{{Caches/}{ProgrammingTools/}{VirtualMemory/}{Market/}{HWParallelismLevel/}
  {SWParallelismLevel/}{VirtualMemory/}}



\begin{document}
\begin{frame}
\titlepage
\end{frame}
\section{Multicore Market}
\Slide{Market/MotivatingExample1}
\Slide{Market/MotivatingExample2}
\Slide{Market/MotivatingExample3}
\Slide{Market/MotivatingCode1}
\Slide{Market/MotivatingCode2}
\Slide{Market/MotivatingCode3}
\begin{frame}
  \begin{multicols}{2}
{\small    \tableofcontents[
% currentsubsection, 
% hideothersubsections, 
% sectionstyle=show/hide, 
% subsectionstyle=show/shaded, 
]}
  \end{multicols}
\end{frame}
\section{HW Parallelism Level}
\Slide{HWParallelismLevel/MultipleLevel}
\Slide{HWParallelismLevel/ILP}
\Slide{HWParallelismLevel/VLIW}
\Slide{HWParallelismLevel/uArch}
% \Slide{HWParallelismLevel/OOO}
% \Slide{HWParallelismLevel/Hyperthreading}
\Slide{HWParallelismLevel/MultiCore}
\Slide{HWParallelismLevel/MultiCPU}
\Slide{HWParallelismLevel/Cluster}
\Slide{HWParallelismLevel/Cluster2}
\Slide{HWParallelismLevel/MPSoC}
\Slide{HWParallelismLevel/GPU}

\section{SW Parallelism Level}
\Slide{SWParallelismLevel/Compiler}
\Slide{SWParallelismLevel/Thread}
\Slide{SWParallelismLevel/Thread-Example}
\Slide{SWParallelismLevel/MPI}
\Slide{SWParallelismLevel/MPI-Example}
\Slide{SWParallelismLevel/OpenMP}
\Slide{SWParallelismLevel/OpenCL}
\Slide{SWParallelismLevel/OpenCL-Example}
\Slide{SWParallelismLevel/CUDA}

\end{document}

% Local Variables: ***
% compile-command:"make -k" ***
% End: ***
